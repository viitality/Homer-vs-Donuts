\newpage
\section{Mode de travail}

\vspace{0.6cm}

\subsection{Espace de travail}
Nous avons très rapidement mis en place un espace de travail pour être au plus efficace :

\begin{itemize}
  \item[$\bullet$]création d'un drive commun pour se partager des documents. Nous avons déjà un document qui nous servira à noter à chaque séance nos avancées et ainsi cela permet d'une part de se rendre compte de où nous en sommes et d'une autre part nous permettra vers la fin du projet de garder les dates et l'ordre de nos avancées ;
  \item[$\bullet$] création d'un serveur discord afin de faciliter la discussion et le partage d'information rapidement comme des liens de sites internet ou autres. De plus, si besoin nous pourrons alors faire des réunions à distance et parler dans des salons vocaux facilement ; 
  \item[$\bullet$] création d'un espace git afin de partager nos codes. 
  (\href{https://github.com/viitality/Homer-vs-Donuts}{Lien Github})
  
\end{itemize}

\vspace{0.6cm}

\subsection{Organisation du travail}

Nous sommes mis d'accord avec le groupe pour se retrouver chaque mercredi après-midi en présentiel afin d'avancer sur le projet et éventuellement discuter de nos avancées personnelles concernant le travail. Néanmoins, en cas de d'indisponibilité, nous organiserons des réunions Zoom en distanciel.
\\

Pour la mise en place d'un bon environnement de travail, nous suivons les valeurs suivantes afin de mener à bien les objectifs et d'avoir une bonne entente de groupe :
\begin{itemize}
\item[$\bullet$] motivation, c'est un sujet qui nous intéresse et on doit se donner à fond pour le mener à bien ;
\item[$\bullet$] esprit d’équipe, c'est un travail de groupe avant tout comme on pourrait avoir plus tard dans la vie professionnelle alors savoir travailler en équipe sera inévitable ;
\item[$\bullet$] communication, nécessaire pour se comprendre et partager les informations ;
\item[$\bullet$] efficacité, comme avait dit Bill Gates : "Je choisis une personne paresseuse pour un travail difficile, car une personne paresseuse va trouver un moyen facile de le faire", c'est-à-dire que notre travail doit être clair et efficace (passer des heures sur un problème ne signifie pas qu'il sera bien résolu) ;
\item[$\bullet$] confiance, puisque l'on va se répartir des tâches, il faut apprendre à faire confiance à ses camarades pour travailler efficacement en équipe ;
\item[$\bullet$] respect, cela paraît évidant mais on doit respecter notre encadrante mais aussi chaque personne du projet ;
\item[$\bullet$] etc.
\end{itemize}