\documentclass[a4paper]{report}

%====================== PACKAGES ======================
\usepackage[top=2cm, bottom=2cm, left=2cm, right=2cm]{geometry}
\usepackage[utf8]{inputenc}
\usepackage[T1]{fontenc}
\usepackage{amsmath,amsfonts,amssymb}

\usepackage[french]{babel}
%pour gérer les positionnement d'images
\usepackage{float}
\usepackage{array}
\usepackage{enumitem}
\usepackage{multirow}
\usepackage{mathpazo}
\usepackage{multicol}
\usepackage{nicefrac}
\usepackage{graphicx, epsfig}
\usepackage[colorinlistoftodos]{todonotes}
\usepackage{url}
%pour les informations sur un document compilé en PDF et les liens externes / internes
\usepackage{hyperref}
%pour la mise en page des tableaux
\usepackage{array}
\usepackage{tabularx}
%pour utiliser \floatbarrier
%\usepackage{placeins}
%\usepackage{floatrow}
%espacement entre les lignes
\usepackage{setspace}
%modifier la mise en page de l'abstract
\usepackage{abstract}
%police et mise en page (marges) du document
\usepackage[T1]{fontenc}
\usepackage[top=2cm, bottom=2cm, left=2cm, right=2cm]{geometry}
%Pour les galerie d'images
\usepackage{subfig}
\usepackage{placeins}

%====================== INFORMATION ET REGLES ======================

%rajouter les numérotation pour les \paragraphe et \subparagraphe
\setcounter{secnumdepth}{4}
\setcounter{tocdepth}{4}

\hypersetup{							% Information sur le document
pdfauthor = {AARAB Maryam,
			BENELKATER Mohamed,
			CORRIOU Alexandre,
    		PREVOT Alexia},			% Auteurs
pdftitle = {PROCESSUS DE DÉCISION MARKOVIEN
Un cas particulier : Homer VS Donuts},			% Titre du document
pdfsubject = {Mémoire de Projet},		% Sujet
pdfkeywords = {Tag1, Tag2, Tag3, ...},	% Mots-clefs
pdfstartview={FitH}}					% ajuste la page à la largueur de l'écran
%pdfcreator = {MikTeX},% Logiciel qui a crée le document
%pdfproducer = {}} % Société avec produit le logiciel

%======================== DEBUT DU DOCUMENT ========================

\begin{document}

%régler l'espacement entre les lignes
\newcommand{\HRule}{\rule{\linewidth}{0.5mm}}

%page de garde
\begin{titlepage}
\begin{center}

% Upper part of the page. The '~' is needed because only works if a paragraph has started.
\includegraphics[width=0.35\textwidth]{./Logo_Polytech_Sorbonne}~\\[1cm]
\includegraphics[width=0.35\textwidth]{./Logo_Sorbonne_Université}~\\[1cm]

\textsc{\Large }\\[0.5cm]

% Title
\HRule \\[0.4cm]

{\huge \bfseries Rapport d'avancement\\
\large
Projet : PROCESSUS DE DÉCISION MARKOVIEN\\
Un cas particulier : Homer VS Donuts \\[0.4cm] }

\HRule \\[1.5cm]

\includegraphics[width=0.35\textwidth]{./homer}~\\[1cm]

% Author and supervisor
\begin{minipage}{0.4\textwidth}
\begin{flushleft} \large
AARAB \textsc{Maryam}\\
BENELKATER \textsc{Mohamed}\\
CORRIOU \textsc{Alexandre}\\
PREVOT \textsc{Alexia}
\end{flushleft}
\end{minipage}
\begin{minipage}{0.4\textwidth}
\begin{flushright} \large
\emph{Encadrant:} \\
Jeanne \textsc{Barthélemy}\\
\emph{MAIN3} \\
2020/2021
\end{flushright}
\end{minipage}

\vfill

% Bottom of the page
{\large \today}

\end{center}
\end{titlepage}

%page blanche
\newpage
~
%ne pas numéroter cette page
\thispagestyle{empty}

\Large
\tableofcontents
%\thispagestyle{empty}
%\setcounter{page}{0}
%ne pas numéroter le sommaire

%\newpage

%espacement entre les lignes d'un tableau
\renewcommand{\arraystretch}{1.5}

%====================== INCLUSION DES PARTIES ======================

~
\thispagestyle{empty}
%recommencer la numérotation des pages à "1"
\setcounter{page}{0}
\newpage
\makeatletter\@addtoreset{section}{part}\makeatother
\renewcommand{\thesection}{\arabic{section}}
\normalsize
\part{Pésentation}

\section{Le sujet}

\section{Démarche adoptée}


\newpage
\section{Mode de travail}

\vspace{0.6cm}

\subsection{Espace de travail}
Nous avons très rapidement mis en place un espace de travail pour être au plus efficace :

\begin{itemize}
  \item[$\bullet$]création d'un drive commun pour se partager des documents. Nous avons déjà un document qui nous servira à noter à chaque séance nos avancées et ainsi cela permet d'une part de se rendre compte de où nous en sommes et d'une autre part nous permettra vers la fin du projet de garder les dates et l'ordre de nos avancées ;
  \item[$\bullet$] création d'un serveur discord afin de faciliter la discussion et le partage d'information rapidement comme des liens de sites internet ou autres. De plus, si besoin nous pourrons alors faire des réunions à distance et parler dans des salons vocaux facilement ; 
  \item[$\bullet$] création d'un espace git afin de partager nos codes. 
  (\href{https://github.com/viitality/Homer-vs-Donuts}{Lien Github})
  
\end{itemize}

\vspace{0.6cm}

\subsection{Organisation du travail}

Nous sommes mis d'accord avec le groupe pour se retrouver chaque mercredi après-midi en présentiel afin d'avancer sur le projet et éventuellement discuter de nos avancées personnelles concernant le travail. Néanmoins, en cas de d'indisponibilité, nous organiserons des réunions Zoom en distanciel.
\\

Pour la mise en place d'un bon environnement de travail, nous suivons les valeurs suivantes afin de mener à bien les objectifs et d'avoir une bonne entente de groupe :
\begin{itemize}
\item[$\bullet$] motivation, c'est un sujet qui nous intéresse et on doit se donner à fond pour le mener à bien ;
\item[$\bullet$] esprit d’équipe, c'est un travail de groupe avant tout comme on pourrait avoir plus tard dans la vie professionnelle alors savoir travailler en équipe sera inévitable ;
\item[$\bullet$] communication, nécessaire pour se comprendre et partager les informations ;
\item[$\bullet$] efficacité, comme avait dit Bill Gates : "Je choisis une personne paresseuse pour un travail difficile, car une personne paresseuse va trouver un moyen facile de le faire", c'est-à-dire que notre travail doit être clair et efficace (passer des heures sur un problème ne signifie pas qu'il sera bien résolu) ;
\item[$\bullet$] confiance, puisque l'on va se répartir des tâches, il faut apprendre à faire confiance à ses camarades pour travailler efficacement en équipe ;
\item[$\bullet$] respect, cela paraît évidant mais on doit respecter notre encadrante mais aussi chaque personne du projet ;
\item[$\bullet$] etc.
\end{itemize}

\part{Début du travail, d'un point de vue théorique et/ou numérique}


\newpage
\section{Conclusion}

\vspace{1 cm}

Ainsi, ce projet fait intervenir de nombreuses notions de mathématiques, de machine learning, d'inforamtique et autres. Notre objectif principale pour le moment sera de chercher un moyen d'apprentissage pour résoudre notre problème et trouver le chemin le plus optimisé afin maximiser le score. Par la suite nous pourrons améliorer les interfaces graphiques pour afficher le score par exemple et enfin en fonction de notre avancement, nous pourrons réfléchir à un modèle plus complexe de ce jeu comme par exemple le positionement aléatoire des agents au début de partie.


%récupérer les citation avec "/footnotemark"
\nocite{*}

%choix du style de la biblio
\bibliographystyle{plain}
%inclusion de la biblio
\bibliography{Bibliographie.bib}
%voir wiki pour plus d'information sur la syntaxe des entrées d'une bibliographie


\end{document}